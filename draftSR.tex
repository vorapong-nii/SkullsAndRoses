%% Game Informatics Workshop 2013
%% A complete game analysis of "Skull and Roses"
%% V1.0
%% 2013/05/29
%% Using bare_conf.tex by Michael Shell

\documentclass[conference]{IEEEtran}

\usepackage{amsmath,amsfonts,amssymb,stmaryrd,fancyhdr,plain,mathrsfs,fixltx2e,theorem}

% Some very useful LaTeX packages include:
% (uncomment the ones you want to load)


% *** CITATION PACKAGES ***
%
%\usepackage{cite}
% cite.sty was written by Donald Arseneau
% V1.6 and later of IEEEtran pre-defines the format of the cite.sty package
% \cite{} output to follow that of IEEE. Loading the cite package will
% result in citation numbers being automatically sorted and properly
% "compressed/ranged". e.g., [1], [9], [2], [7], [5], [6] without using
% cite.sty will become [1], [2], [5]--[7], [9] using cite.sty. cite.sty's
% \cite will automatically add leading space, if needed. Use cite.sty's
% noadjust option (cite.sty V3.8 and later) if you want to turn this off.
% cite.sty is already installed on most LaTeX systems. Be sure and use
% version 4.0 (2003-05-27) and later if using hyperref.sty. cite.sty does
% not currently provide for hyperlinked citations.
% The latest version can be obtained at:
% http://www.ctan.org/tex-archive/macros/latex/contrib/cite/
% The documentation is contained in the cite.sty file itself.




% correct bad hyphenation here
\hyphenation{op-tical net-works semi-conduc-tor}

\newcommand{\sr}{S\&{}R}
\newtheorem{thm}{Theorem}
\newtheorem{pry}{Property}


\begin{document}
%
% paper title
% can use linebreaks \\ within to get better formatting as desired
\title{A complete game analysis of ``Skull and Roses"}


% author names and affiliations
% use a multiple column layout for up to three different
% affiliations
\author{\IEEEauthorblockN{Baffier, Jean-François}
\IEEEauthorblockA{Graduate School of \\Information Science and Technology \\
The University of Tokyo\\
Email: jf.baffier@is.s.u-tokyo.ac.jp}
\and
\IEEEauthorblockN{Gragera Aguaza, Alonso J.}
\IEEEauthorblockA{Graduate School of \\Information Science and Technology \\
The University of Tokyo\\
Email: alonso@is.s.u-tokyo.ac.jp}
\and
\IEEEauthorblockN{Suppakitpaisarn,Vorapong}
\IEEEauthorblockA{National Institute of Informatics\\
Email: vorapong@nii.ac.jp}}



% make the title area
\maketitle


\begin{abstract}
%\boldmath
Skull and Roses is a game of bluff played with 2 kind of cards: \emph{skulls} and \emph{roses}. The players successively place hidden cards on their own pile till one of them bets.
The bet consists in announcing how many roses you think that you can reveal without encountering a skull in the process. After a bet is set, as in poker,
the other players can either pass (definitively) or bet higher, if the challenger meet his bet, he wins one point, otherwise he looses one card.\newline

In this paper, we study in detail... %some properties on the game, give a model for the general case and also provide results for the lower and upper bounds of the game tree and \dots .\newline
\end{abstract}


% no keywords




% For peer review papers, you can put extra information on the cover
% page as needed:
% \ifCLASSOPTIONpeerreview
% \begin{center} \bfseries EDICS Category: 3-BBND \end{center}
% \fi
%
% For peerreview papers, this IEEEtran command inserts a page break and
% creates the second title. It will be ignored for other modes.
\IEEEpeerreviewmaketitle



\section{Introduction}
The \emph{Skull \& Roses} game was created by Hervé \textsc{Marly}, illustrated by Rose \textsc{Kipik} and is edited by \emph{lui-même} since $2011$ -- in German, French and English as the moment this paper was submitted. The game received the price of \textsc{as d'or, jeu de l'année, Cannes 2011}.\\



\section{General model}

\subsection{Notation}
In this paper, we will design by $P$ the number of player, $S$ the original number of skulls, $R$ the original number of roses, and $W$ the number of wins required. A game is defined such that \sr(P,S,R,W) \newline

%An \emph{instance} (round) is defined by the state of each player and the current first player. Given an instance $I$, $r_i$, $s_i$, and $w_i$ refers to the current number of roses, skulls, and wins of player $i$. Then the instance $I$ is defined by $I=|\left(\lbrace (r_i,s_i,w_i)\rbrace, first\right)$.\newline

%A situation is simply the list of cards already played at some point of an instance. Let $c_i$ refers to the list of cards played by player $i$. Then a situation $Sit$ is simply defined by $Sit={c_i|i\in\llbracket 1;n\rrbracket}$.\newline

%The aim of this study is to define the best strategy $Strat$ for a given player, through a full game $G$.

\subsection{Formal rules}
Uncomment %http://www.skull-and-roses.com/pdf/Skull_rules_Us.pdf



\subsection{Representation}
?


\section{Properties}

\subsection{Pile size difference}

\begin{pry}[Pile size difference]
In any situation inside a round, the difference between any pair of player's piles size cannot be more than $1$.
\end{pry}

\subsection{Bet tree size}

\begin{thm}[Bet tree size]
Let the number of cards played $c$ and the actual number of players $p$ for the current situation, the number of leaves of the bet tree is:
\[ 
\sum^c_{j=1} \binom{c}{j} \binom{p-3+j}{j-1}_{.}
\]
\end{thm}





% \subsection{Hash function}
% %For a given situation $Sit$, we may use some hash function $h$ such as $h(RS|SR|R|S)=RS\varnothing SR\varnothing R\varnothing S)$.
% 
% %From a player $i$ point of view, $h_i(RS|X|Y|Z)=h_i(RS|Z|X|Y)=\cdots$.
% 
% We can define two hashing functions $H_\sigma$ and $H_\rho$. $H_\sigma$ is such that we can delete any card of a player card list that precedes the last skull. For example, $H_\sigma(RS|SR|R|S)=H_\sigma(S|SR|R|S)=H_\sigma(RRRSSSRRRSS|SR|R|S)$.
% 
% The second hashing, $H_\rho$ is similar, but conserves the number of cards preceding the last skull. It can be an important information depending of the strategies considered.


%\subsection{Function related to bet tree}
%First, we define a function $f$ such that, for any strategy $Strat$ and any situation $Sit$, $f(Strat,Sit)$ returns a unique bet tree.
%
%Secondly, $g$ is define from the set of all bet trees to $\left[ 0,1 \right] $. It returns the probability of "winning" from an "evaluated" bet tree.

%\subsection{Non-sense moves}
%Subsection text here.

\section{Boundaries}
\subsection{Game length}

\subsubsection{Minimum}
\ \\The minimum number of moves for a \sr(P,S,R,W) game is: 
\[
2 \cdot \min \left( W,(P-1)(R+S)\right)
\]

This is correspond with the case where the first player wins all the challenges in a row and in his first move or all players lose all the challenges. 
\\

\subsubsection{Maximum}
\ \\The minimum number of moves for a \sr(P,S,R,W) game is: 
\[ 
(2M-1)(W-1)P + (M^2 - MP) + (P^2 - P)
\]

\[ 
(2M-1)(W-1)P + (M^2 + P^2 - (M-1)P) 
\]

This is correspond with the case where all players wins W-1 the challenges in a row, then all players lose R+(S-1) cards and at the end they get eliminated one by one.
\\


%\subsubsection{Upper bound}
%\ \\The minimum number of moves for a \sr(P,S,R,W) game is: 
%\[ 
%(2M-1)(W-1)P + (M^2 - P^2) + (P^2 - 1)
%\]
%
%\[ 
%(2M-1)(W-1)P + (M^2 - 1)
%\]
%
%This is correspond with the case where all players wins W-1 the challenges in a row, then all players lose R+(S-1) cards and at the end they get eliminated one by one.
%\\

%\subsubsection{Upper bound}
%The most obvious upper bound is:
%
%\[ (2M-1)(W-1)P+2M^2\]
%
%where $M=(R+S)P$


\subsection{State-space size}

\subsubsection{Lower bound}
\ \\The most obvious lower bound is:
$$((S+1) \cdot (R+1) \cdot (W))^P + (((S+1) \cdot (R+1))^P \cdot (W)^{P-1})$$


%\subsubsection{Lower bound}
%$$P \cdot (S+1)^P \cdot (R+1)^P \cdot (W+1)^P$$
%
%$$P \cdot ((S+1) \cdot (R+1) \cdot (W+1))^P$$

\subsubsection{Upper bound}
\ \\The most obvious upper bound is:

$$P \cdot (S+1)^P \cdot (R+1)^P \cdot (W+1)^P \cdot P \cdot (S+R) \cdot 2^P \cdot (S+R) \cdot (2 \cdot (S+1) \cdot \binom {S+R} {R})^P$$

$$  ((S+R) \cdot P)^2 \cdot (2^2 \cdot(S+1)^2 \cdot (R+1) \cdot (W+1) \cdot \binom {S+R} {R})^P$$

%\subsubsection{Upper bound}
%The most obvious upper bound is:
%
%$$P \cdot (S+1)^P \cdot (R+1)^P \cdot (W+1)^P \cdot P \cdot (S+R) \cdot 2^P \cdot ((S+1) \cdot (R+1) \cdot \binom {S+R} {R})^P$$
%
%$$  (S+R) \cdot P^2 \cdot (2 \cdot(S+1)^2 \cdot (R+1)^2 \cdot (W+1) \cdot \binom {S+R} {R})^P$$



\subsection{Game tree size}

\subsubsection{Lower bound}
\ \\The most obvious upper bound is:

$$3^{\min(W,(R+S)(P-1))}$$

\ \\

\subsubsection{Upper bound}
\ \\The most obvious upper bound is:

$$(2P)^{((S+R)P+1)(WP+(S+R)P)}$$

\ \\

\section{Complexity}

\subsection{From POS CNF to \sr(2,S,R,1)}
\sr(2,m,o,1) is PSpace-complete \newline

\subsection{From \sr(2,S,R,1) to POS CNF}

\subsection{From \sr(2,S,R,1) to \sr(P,S,R,W)}
\sr(n,m,o,p) is at least as difficult as \sr(2,m,o,1) \newline

\ \\

\section{Numeric results}

\begin{center}
  \begin{tabular}{ | l || c | c | c | }
    \hline
    Game & Game length & State-space size & Tree size \\ \hline \hline
    $\sr(3,1,3,2)$ & $183$ & $2.5\cdot10^{8}$ & $3$ \\ \hline
    $\sr(4,1,3,2)$ & $328$ & $8.7\cdot10^{10}$ & $3$ \\ \hline
    $\sr(5,1,3,2)$ & $515$ & $2.6\cdot10^{13}$ & $3$ \\ \hline
    $\sr(6,1,3,2)$ & $774$ & $7.2\cdot10^{15}$ & $3$ \\ \hline
    $\sr(7,1,3,2)$ & $1015$ & $1.9\cdot10^{18}$ & $3$ \\ \hline
    $\sr(8,1,3,2)$ & $1328$ & $4.7\cdot10^{20}$ & $3$ \\ \hline
    $\sr(9,1,3,2)$ & $1683$ & $1.1\cdot10^{23}$ & $3$ \\ \hline
    $\sr(10,1,3,2)$ & $2080$ & $2.7\cdot10^{25}$ & $3$ \\ \hline
    $\sr(11,1,3,2)$ & $2519$ & $6.3\cdot10^{27}$ & $3$ \\ \hline
    $\sr(12,1,3,2)$ & $3000$ & $1.4\cdot10^{30}$ & $3$ \\ \hline
  \end{tabular}
\end{center}

\begin{center}
  \begin{tabular}{ | l || c | c | c | c | }
    \hline
    Game & Average game length & Average branching factor & State-space size celling & Tree size celling \\ \hline \hline
    Tic-Tac-Toe & $9$ & $4$ & $10^{3}$ & $10^{5}$ \\ \hline
    $\sr(3,1,3,2)$ & ? & ? & $10^{9}$ & $3$ \\ \hline
    Connect Four & $36$ & $4$ & $10^{13}$ & $10^{21}$ \\ \hline
    $\sr(6,1,3,2)$ & ? & ? & $10^{16}$ & $3$ \\ \hline
    Backgammon & $55$ & $250$  & $10^{20}$ & $10^{144}$ \\ \hline
    $\sr(12,1,3,2)$ & ? & ? & $10^{31}$ & $3$ \\ \hline
    Chess & $80$ & $36$ & $10^{47}$ & $10^{123}$ \\ \hline
    
  \end{tabular}
\end{center}


%\section{Applications}


%\section{Conclusion}
%Skull and Roses is quite challenging by mixing many concepts of game theory, including hidden games, bluff, dependant and successive instances and long term strategy.\\

%Beside its own merits as game, and because of the various aspects of the interactions covered by \sr \ we think that many applications can be found, as in economic or social sciences.\\



% conference papers do not normally have an appendix


% use section* for acknowledgement
%\section*{Acknowledgment}


%The authors would like to thank...





% trigger a \newpage just before the given reference
% number - used to balance the columns on the last page
% adjust value as needed - may need to be readjusted if
% the document is modified later
%\IEEEtriggeratref{8}
% The "triggered" command can be changed if desired:
%\IEEEtriggercmd{\enlargethispage{-5in}}

% references section

% can use a bibliography generated by BibTeX as a .bbl file
% BibTeX documentation can be easily obtained at:
% http://www.ctan.org/tex-archive/biblio/bibtex/contrib/doc/
% The IEEEtran BibTeX style support page is at:
% http://www.michaelshell.org/tex/ieeetran/bibtex/
%\bibliographystyle{IEEEtran}
% argument is your BibTeX string definitions and bibliography database(s)
%\bibliography{IEEEabrv,../bib/paper}
%
% <OR> manually copy in the resultant .bbl file
% set second argument of \begin to the number of references
% (used to reserve space for the reference number labels box)

%\begin{thebibliography}{1}
%
%\bibitem{IEEEhowto:kopka}
%H.~Kopka and P.~W. Daly, \emph{A Guide to \LaTeX}, 3rd~ed.\hskip 1em plus
%  0.5em minus 0.4em\relax Harlow, England: Addison-Wesley, 1999.
%
%  
%  
%\end{thebibliography}




% that's all folks
\end{document}
